\section{Introduction}
Vehicular networks  (VANET) as all wireless networks can be easily subject to any kinds of security  attacks  then , it need an appropriate security achitecture. One of the principale features is to ensure the integrity of messages exchanged between vehicles and the authentication of the sender.\\
Our project was to implement a simple frame work for the simulation of secure messages exchange in VANET. The simulator is essentially based on  signing  messages  before send its through the network and to able to verify  the integrity of any messages received from the VANET and also verify check the authentiction of the sender.\\
The techniques used to provide these security features are essentially based on two mains approachs provided by\cite{calandriello} which are :
\begin{itemize}
\item BaseLine Pseudonyms, a pseudonym is a public certified key,
the basic approach to privacy is based on periodically changing the psuedonym.
\item Hybrib Scheme,is the combination of two approachs coupling the pseudonym generation and the Group signature scheme
\\
\url{http://www.sigmobile.org/workshops/vanet2007/slides/3.pdf}
\end{itemize}
These two techniques during the implatations has been subject to somes otpimization.
\\
\\