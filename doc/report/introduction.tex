\section{Introduction}
A Vehicular Ad-Hoc Network, or VANET, is a form of Mobile ad-hoc network, to provide communications among nearby vehicles and between vehicles and nearby fixed equipment, usually described as roadside equipment. It will contribute to safer and more efficient roads by providing timely information to drivers and concerned authorities.
As all wireless networks, VANET can be easily subject to any kinds of security  attacks then, for ensure a good security, VANET needs an appropriate security architecture that should be based on different security aspects. One of the most important security aspect on which we focused our work is Authentication and Key management.\\
Our project was to implement a simple frame work for the simulation of secure messages exchange in VANET. The simulator is essentially based on  signing  messages before send them in network area, to be able to verify the integrity of any messages received from the network area. All the messages are sent in broadcast way in a limit area\\
The techniques used to provide these security features are essentially based on two mains approaches provided
 by\cite{calandriello} which are :
\begin{itemize}
\item BaseLine Pseudonyms,\\
In this aspect a considerable amount of certified public key called pseudonym  will stored in a vehicle, in order the ensure the privacy of the sender it will continuously change the public key certificate or pseudonym to sign a given amount of messages or to sign for a limited time this to avoid the sender to be retrieved by some receivers from a subsequent signed messages. These pseudonym should issued by a well known Certificated Authority in order to provide authentication.

\item Hybrib Scheme\\
This technique combines two approaches, coupling the pseudonym generation and the Group signature scheme.
Instead of stored the pseudonyms like in the previous case, the pseudonym  is generated on the fly when it want to send a message, it has to be sign by a group public key provided to any vehicle by some group manager in order to warranty the authentication between user of the same group.
\\
\url{http://www.sigmobile.org/workshops/vanet2007/slides/3.pdf}
\end{itemize}
These two techniques during the implatations has been subject to somes otpimization.
\\
\\