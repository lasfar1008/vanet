\section{Introduction}
A Vehicular Ad-Hoc Network, or VANET, is a form of Mobile ad-hoc network, to provide communications among nearby vehicles and between vehicles and nearby fixed equipment, usually described as roadside equipment. It will contribute to safety and more efficient roads by providing information to drivers and concerned authorities around street condition, traffic problems and others.
As all wireless networks, VANET can be easily subject to any kinds of security  attacks then, for ensure a good security, VANET needs an appropriate security architecture that should be based on different security aspects like: Threat model, Authentication and key management, Privacy, Secure positioning. We focused our work in on the Authentication and Key management.\\
Our project was to implement a simple framework for the simulation of secure messages exchange in VANET. The simulator is essentially based on  signing  messages before send them in network area, to be able to verify the integrity of any messages received from the network area. All the messages are sent in broadcast way in a limited area\\
Techniques used to provide these security features are essentially based on two mains approaches provided by\cite{calandriello} which are :
\begin{itemize}
\item BaseLine Pseudonyms,\\
In this aspect a considerable amount of certified public key called pseudonym will stored in a vehicle, in order the ensure the privacy of the sender it will continuously change the public key certificate or pseudonym to sign a given amount of messages or to sign for a limited time this to avoid the sender to be retrieved by some receivers from a subsequent signed messages. These pseudonym should issued by a well known Certificated Authority in order to provide authentication.
\item Hybrib Scheme\\
This technique combines two approaches, coupling the pseudonym generation and the Group signature scheme.
Instead of stored the pseudonyms like in the previous case, the pseudonym  is generated on the fly when the vehicle wants to send a message, it has to be sign by a group public key provided to any vehicle by a Certificate Authority that operate as a group manager in order to guaranty the authentication between users assuming that all legitimate vehicles belongs to the same group.
\\
\url{http://www.sigmobile.org/workshops/vanet2007/slides/3.pdf}
\end{itemize}
These two techniques during the implementations has been subject to some optimizations.
\section{Framework Introduction}
\subsection{Security Implementations}\label{sec:CryptographyOverhead}
Before starting with  a real implementation of VANET Simulator we want to focus on a security features that has been chosen. This technology is focused around four main task: privacy, authentication, integrity and no-repudiation; and we can note that these properties are related to asymmetric cryptography.\\ 
We have two  principals techniques for asymmetric cryptography that can provide these requires properties : RSA and Elliptic curves. At the beginning of this project, we have implemented the first one, RSA with 1024 bit of security level because almost useful functionality are already available for Java Platform. After few tests performed we meet a lot of problem related to the implementation of the solution. First of all we have a real good time responses, see table \ref{tab:RSAVelocity} at page \pageref{tab:RSAVelocity}, for sign and verify and authority verification but the big problems where at the trasmission level caused by packet dimensions.
\begin{table}[!ht]
	\centering
	\caption{RSA Speed Analysis}
	\begin{tabular}{|c|c|c|c|c|c|c|}
	\hline\hline 
	\textbf{Provider} & \textbf{Sign} & \textbf{Verify} & \textbf{CA verification} & \textbf{sign/s} & \textbf{verify/s}  & \textbf{CA verify/s}\\
	\hline
	Java2 1.6u14 & 0.009284s & 0.000521s & 0.000525s & 107.7 & 1919 & 1904 \\
	\hline
	\hline     %inserts single line 
 	\end{tabular} 
	\label{tab:RSAVelocity}
\end{table}
\subsection{From RSA to EC}
If we consider to send on network beacons of 200 bytes with RSA technique we have an enormous cryptography overhead, see table the following table \ref{tab:RSAOverHead} at page \pageref{tab:RSAOverHead}.
\begin{table}[!ht]
	\centering
	\caption{RSA Cryptography Overhead}
	\begin{tabular}{|c|c|c|}
	\hline\hline 
	\textbf{Payload} & \textbf{Signature} & \textbf{Certificate}\\
	\hline
	200 & 128 & 1029 \\
	\hline
	\hline     %inserts single line 
 	\end{tabular} 
	\label{tab:RSAOverHead}
\end{table}
The real problem with RSA is the signature length and more precisely the  certificate length. If we want 1024 bit of security level we cannot modify signature length but we can modify the  certificate length. To do that, we have constructed a X509 striped certificate for reduce dimension of packet removing all useless information from the  certificate. But the length of the certificate reduced dimension to 750 bytes, that still too big. The real problem is length of public key inside certificate and we can not modify that unless we reduce security level.\\
At this point we have decided to use elliptic curves according to data provided in \cite{calandriello}. Consider the same data, for payload, around 200 bytes with 163 bit of security level using elliptic curves we have very little certificates and public key. The longest message in the network is around 500 bytes, during this monography we analyze optimizations for reduce packet dimension to 250 bytes. See table \ref{tab:CryptographyOverhead} at page \pageref{tab:CryptographyOverhead} for more details on cryptography overhead.
\begin{table}[!ht]
	\centering
	\caption{Elliptic curves cryptography overhead}
	\begin{tabular}{|c|c|c|}
	\hline\hline 
	\textbf{Payload} & \textbf{Signature} & \textbf{Certificate}\\
	\hline
	$200$ &	$48$ & $234$\\
	\hline
	\hline     %inserts single line 
 	\end{tabular}
	\label{tab:CryptographyOverhead}
\end{table}
\subsection{Role of Certification Authority and CRL}
Certification authority and CRL for VANET Simulator is actually a point of discussion \cite{calandriello}, in particular the positioning of CA and CRL is really difficult for vehicular networks. VANET Simulator implement only one CA and not use certificate revocation list. 
\subsection{Vehicle movement's}
The simulator is also able to simulate vehicle movements, it's implementation is really simple because simulator do not consider the collisions of two or more vehicles or particular position for a given vehicle. VANET simulator move all the vehicles on wireless area according to the velocity information, computing how many cells of WIFI it have to jump. Vehicle movements is only in horizontal direction, from left to right, no others movements are considered.\\
If a vehicle go out of the wireless area it will remain there for one cycle, the vehicle stop any transmission and reception. After a cycle time it will re-enter into WIFI with 180 degree of rotation, \textit{y} coordinates do not change.