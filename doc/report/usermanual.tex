\section{Documentation}
\subsection{User Manual}
In this part of monography we explain the possibilities offered by configurations for using the Vanet Simulator in all of it parts. In particular the main features offered by simulator are changing the security implementation passing by \baseline and \hybrid implementations and configure vehicles on the road, the number of beacons sent during the simulations and modifying logging system for understanding results of simulation.\\
The Vanet Simulator is completely configurable modifying it's configurations files under the folder \textit{properties} in the root directory of simulator.
\subsubsection{Base Configurations}\label{usermanual:baseconfiguration}
The base configurations provide modifications in the core of Vanet Simulator, in particular you can change the \textit{base.properties} file for change core properties like beacons sent, security implementations and others base properties.
If you open the configuration file you see:
\begin{verbatim}
#Max speed in km/h
max_speed = 140
#Max 802.11 cover in meters
wifi_cover = 200
#Access Point broadcast point
server_broadcast_point = 127.255.255.255
#Server Port
server_port = 55055
#Beacons/sec
beacons_sec = 0.1
#If you want no moves of vehicles
no_moves = true
#choose simulator BP or GS
simulator = bp
#max certificate validity into area in seconds
maxCertificateValidityTime = 33
#Reattach certificate every tot beacons
reattachCertificate = 5
#MYSQL properties
mysql_host=127.0.0.1
mysql_username=root
mysql_password=
mysql_database=vanet
#Define the log system
#  0 MYSQL log
#  1 File log
#  2 StdOut log
logSystem=0
\end{verbatim}
For detailed information you can see table \ref{tab:BaseConfiguration} at page \pageref{tab:BaseConfiguration}.
\begin{table}[!ht]
	\centering
	\caption{Base Configuration specifications}
	\begin{tabular}{|c|c|c|}
	\hline\hline 
	\textbf{Property Name} & \textbf{Property Translation} & \textbf{Property Type} \\
	\hline
	max\_speed & The maximum speed of vehicles & int \\
	\hline
	wifi\_cover & The maximum wireless area coverage & int \\
	\hline
	server\_broadcast\_point & The broadcast node for send messages & string \\
	\hline
	server\_port & The port for receive messages & int \\
	\hline
	beacons\_sec & The number of beacons sent in one second & float \\
	\hline
	no\_moves & Lock vehicles into the map & boolean \\
	\hline
	simulator & The simulator which you want use. & string\\
	{} & BP for baseline implementations. & {} \\
	{} & GS for group signature implementation & {} \\
	\hline
	maxCertificateValidityTime & The maximum time for certificate validity & int \\
	\hline
	reattachCertificate & Reattach the certificate every tot beacons & int \\
	\hline
	mysql\_host & The host for mysql & string \\
	\hline
	mysql\_username & Username for authenticate into mysql & string \\
	\hline
	mysql\_password & Password for authenticate into mysql & string \\
	\hline
	mysql\_database & Database to use & string \\
	\hline
	logSystem & The log system which you want use. & int\\
	{} & 0 for mysql log system. & {} \\
	{} & 1 for log data into a files & {} \\
	{} & 2 for log data on console & {} \\
	\hline
	\hline     %inserts single line 
 	\end{tabular} 
	\label{tab:BaseConfiguration}
\end{table}
\subsubsection{Vehicle Configurations}\label{usermanual:vehicleconfigurations}
The vehicles configurations set the status of roads into the simulator. In particular you can modify the number of vehicles into the road, velocity and initial position.\\
The configuration file for vehicle is XML (eXtensible Markup Language) based and is named \textit{vehicles.xml} and positioned into folder \textit{vehicles}; if you open this file you see:
\begin{verbatim}
<?xml version="1.0" encoding="UTF-8"?>
<Vehicles>
    <Vehicle id="1" speed="100" x="10" y="20" />
    <Vehicle id="2" speed="120" x="20" y="50" />
</Vehicles>
\end{verbatim}
Every tag, excluding root tag, identify a new vehicle with attributes like options, in particular you can modify the vehicle identification number changing the \textit{id} attribute or you can change the vehicle speed modifying the \textit{speed} attribute or the position of the vehicle using the \textit{x} or \textit{y} attributes.
For the \baseline~operating mode you have to link certificates and private keys to each vehicles, for doing that you have to follow instruction in section \ref{usermanual:preloadkey} at page \pageref{usermanual:preloadkey}
\subsubsection{Why many log configurations}
The log system for this application is really difficult, in fact the normal std out log system is too slow and produce conflicts if you send a lot of beacons during sign and verify operation but it's really useful because you can understand immediately what the system are doing in real time, the other methods are a middle solution for see result and velocity during sign and verify and the best solution for velocity but difficult to understand in real time the system but it's useful for post-processing. For this reason we have written three type of log system for use the best method when that are compatible with the simulation.
\subsubsection{Log configuration}\label{usermanual:logconfiguration}
The log system use the \textit{log4j} module for write sensible information of simulator. The system provide three log configurations, on standard output stream, file stream or on MySQL database.\\
The configuration of log system it's really powerful and you can set the level of logging or change the log representation for standard out stream or file stream. The configuration of log system is divided into three file, ones for each method and it's collected into folder \textit{properties} which names \textit{stdout.properties} for standard out, \textit{file.properties} for file stream or \textit{mysql.properties} for MySQL database log system.
\subsubsection{MySQL database configuration}
For using MySQL database log system you have to configure the database before launching the Vanet Simulator. You have to create or import a database with tables definition into MySQL using the \textit{vanet.sql} file under the \textit{properties} directory.\\
For create the database and tables definition you have to enter in you MySQL command line and create a new database using command:
\begin{verbatim}
mysql> CREATE DATABASE vanet;
\end{verbatim}
After this step you have to create a table in the new database using commands:
\begin{verbatim}
mysql> use vanet;
...
mysql>CREATE TABLE IF NOT EXISTS `logs` (
  `log_id` int(11) unsigned zerofill NOT NULL auto_increment,
  `level` varchar(255) NOT NULL,
  `class_name` varchar(255) NOT NULL,
  `method_name` varchar(255) NOT NULL,
  `message` text NOT NULL,
  PRIMARY KEY  (`log_id`),
  KEY `level` (`level`)
) ENGINE=InnoDB DEFAULT CHARSET=latin1 AUTO_INCREMENT=1 ;
...
\end{verbatim}
After this step you have configured the MySQL database for record logs from Vanet Simulator.
\subsubsection{Output Reading}
VANET simulator have a log system for showing information, this lines are leveled on five states:
\begin{description}
	\item [trace] Normal trace operation for follow program execution line
	\item [debug] Debug general information
	\item [warn] Warning, little problem, not critical
	\item [error] Severe error
	\item [fatal] Impossibile to run over, the program terminate immediatly
\end{description}
The simulator can provided log  information's with three ways, directly on your screen using \textit{std out} or write into a new file or store into a database table. All log system show the same information but structured in different ways, a typical std out representation is:
\begin{verbatim}
07:22:22,870 [1632] [Vehicle - 1] (new line for readability reasons)
	DEBUG vanet.Vehicle:105 - (new line for readability reasons)
		Vehicle 1 position, x=37 y=20(new line for readability reasons)
		
07:22:22,870 [1632] [Vehicle - 1] (new line for readability reasons)
	DEBUG vanet.Vehicle:108 - (new line for readability reasons)
		Vehicle 1 send new message(new line for readability reasons)
		
07:22:22,905 [1667] [Vehicle - 2] (new line for readability reasons)
	DEBUG vanet.Vehicle:105 - (new line for readability reasons)
		Vehicle 2 position, x=53 y=50 (new line for readability reasons)
\end{verbatim}
All lines are written with the same logic, in particular start with the time using format: hour, minutes and seconds followed by microseconds. After that into square brackets you have the number of milliseconds elapsed since the start of program, followed by the name of the Java class which execute log, always into square brackets. Then in upper case you have the level of log, followed by the complete name of class followed by double dots and the line of program execution. After that a free text message which inform you on operations, like state changes or errors.\\
Logging into file system follow the same representation and database implementation instead use one table organized in five fields, see table \ref{tab:DBLog} at page \pageref{tab:DBLog}.
\begin{table}[!ht]
	\centering
	\caption{Database table for logging system}
	\begin{tabular}{|c|c|c|c|c|}
	\hline\hline 
	\textbf{Log identification} & \textbf{Level of log} & \textbf{Class name} & \textbf{Method name} & \textbf{Free message}\\
	\hline
	\hline     %inserts single line 
 	\end{tabular} 
	\label{tab:DBLog}
\end{table}
Log identification is an unique number for log entry, other fields is the same of other log methods.
\subsubsection{Read vanet simulator operation from log}
Before closing the output reading we want to explain with an example the normal flow of Vanet Simulator with vehicles on the roads. The simulator is multi-thread and don't use syncronization, for that reason the output could be confused. In this section we log pieces of log output but for readability reasons we remove all information which do not have sense for this section, like timestamp and other not useful parts. In addition long line are splitted into two lines, but we consider for one line only those who start with square brackets.\\
\begin{verbatim}
[Vehicle - 1] DEBUG vanet.Vehicle:105 - Vehicle 1 position, x=37 y=20
[Vehicle - 1] DEBUG vanet.Vehicle:108 - Vehicle 1 send new message
[Vehicle - 2] DEBUG vanet.Vehicle:105 - Vehicle 2 position, x=53 y=50
[Vehicle - 2] DEBUG vanet.Vehicle:108 - Vehicle 2 send new message
[Vehicle - 1] DEBUG vanet.security.BaseLinePseudonyms:128 - Message sent in 
  LONG MODE. Certificate expired use new pseudonymous. Certificate ID: 12884
[Vehicle - 2] DEBUG vanet.security.BaseLinePseudonyms:128 - Message sent in 
  LONG MODE. Certificate expired use new pseudonymous. Certificate ID: 54389
...
...
[Vehicle - 2] DEBUG vanet.security.BaseLinePseudonyms:178 - Message sent in 
  SHORT MODE for certificate. Certificate ID: 54389
\end{verbatim}
In the first 6 line, two vehicles have sent new message and these messages are securized using a new certificate (pseudonym), when we use the word \textit{long} we want identify the attach of certificate at the end of message instead with word \textit{short} when we send a message without certificate, in line with optimization 2 in \cite{calandriello}.\\
The system send message in broadcast and for that reason we receive also our messages and in line:
\begin{verbatim}
[Transceiver For Vehicle] INFO  vanet.Transceiver:123 - Skip auto-verification 
  for message: 54389
\end{verbatim}
the transceiver skip verification of self-messages for reduce computational resource. The same operation is executed by the other vehicle. When a vehicle receive a message securized by another vehicle, check it validity and write result of verification on log
\begin{verbatim}
[Transceiver For Vehicle] INFO  vanet.Transceiver:113 - Message secure: 54389
\end{verbatim}
If a signature is not valid or a vehicle do not have certificate for check a message, write a message insecure log:
\begin{verbatim}
[Transceiver For Vehicle] WARN  vanet.Transceiver:118 - Message insecure: 2804
\end{verbatim}
Another information expressed by logs is around positioning of vehicles, in particular when a vehicle go out of wifi range the system express this condition with a message like this:
\begin{verbatim}
[Vehicle - 1] DEBUG vanet.Vehicle:105 - Vehicle 1 position, x=276 y=20
[Vehicle - 1] INFO  vanet.Vehicle:113 - Vehicle 1 is not in the WIFI area, 
  set x position equal to 0
\end{verbatim}
\subsubsection{Install Vanet Simulator on Windows}
For install Vanet simulator on Windows operating system you can use the installer \textit{setupVanetSimulator.exe} and follow the screen information for complete the setup of application.\\
After install procedure you have to open a new console and go into install directory and send the command
\begin{verbatim}
C:\Program File\Vanet Simulator\>java -jar vanetSimulator.jar
\end{verbatim}
After this command you see the bootstrap procedure and after the system run. The default log operation is the standard out and you can see directly all the informations.\\
The common output on screen are this:
\begin{verbatim}
.:: Bootstrap ::.
Loading base properties
Base properties loaded
Loading vehicles configuration
security/certificates/1/c
security/certificates/2/c
Vehicles configuration loaded
.:: Bootstrap end ::.
\end{verbatim}
\subsubsection{Install Vanet Simulator on generic OS}
For install Vanet Simulator you have to setup all folders and executable jar manually. Create new folder in a point of file system and enter in it, after that copy the content of \textit{build} directory and now you can send the command for start the simulator.
\begin{verbatim}
name@domain$ java -jar vanetSimulator.jar
\end{verbatim}
After this command you see the bootstrap procedure and after the system run. The default log operation is the standard out and you can see directly all the informations.\\
The common output on screen are this:
\begin{verbatim}
.:: Bootstrap ::.
Loading base properties
Base properties loaded
Loading vehicles configuration
security/certificates/1/c
security/certificates/2/c
Vehicles configuration loaded
.:: Bootstrap end ::.
\end{verbatim}
\subsubsection{Add certificates and private keys for \baseline}\label{usermanual:preloadkey}
When the system doing the bootstrap in \baseline~mode research certificate and private keys for doing digital signatures.\\
Security properties are in \textit{security} folder, if you add one vehicle you must attach certificates and private keys for work with \baseline~implementation. For do that you have to create a set of folders and positioning certificates and keys in a right place. Under \textit{security} you have one folder named \textit{certificates} and under that folder you have another one folder for each vehicle named with the \textit{vehicle id} (section \ref{usermanual:vehicleconfigurations} at page \pageref{usermanual:vehicleconfigurations}), create new folder named with unique integer identification. Under this folder you have to create another two folders named \textit{c} for certificates and \textit{p} for private keys; after this steps you have to copy your private keys in folder \textit{p} and certificates in folder \textit{c}. Certificates and private keys must have the same name for link, for example: \textit{sec1.crt $\rightarrow$ sec1.key}.tificates and keys in a right place. Under \textit{security} you have one folder named \textit{certificates} and under that folder you have another one folder for each vehicle named with the \textit{vehicle id} (section \ref{usermanual:vehicleconfigurations} at page \pageref{usermanual:vehicleconfigurations}), create new folder named with unique integer identification. Under this folder you have to create another two folders named \textit{c} for certificates and \textit{p} for private keys; after this steps you have to copy your private keys in folder \textit{p} and certificates in folder \textit{c}. Certificates and private keys must have the same name for link, for example: \textit{sec1.crt $\rightarrow$ sec1.key}.

